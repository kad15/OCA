\section{Décomposition réalisée}

\paragraph{Sous-problème primal (SP($\hat{z}$)) :}
Soit $\hat{z}$ un vecteur binaire quelconque de taille card(H) de composantes $\hat{z}_i$ \textbf{fixé}. Le sous-problème primal à optimiser par rapport aux seules variables $x_{ek}$ est : 

\[v(\hat{z}) =  \min \sum_{e \in E_k}\sum_{k \in K}F_{ek}x_{ek}\]
tel que
\begin{subequations}
    \begin{align}
        \sum_{e \in E_k}x_{ek} = 1, \quad &\forall{k\in K} \label{2a} &\\
        x_{ek} \ge 0, \quad &\forall{k \in K}, \forall{e \in E_k} \label{2b}&\\
        \sum_{e \in E_k:i\in e}x_{ek} \le \hat{z}_i,\quad &\forall{i \in H}, \forall{k\in K}\label{2c}
    \end{align}
\end{subequations}



\smallskip

\paragraph{Sous-problème dual (SD($\hat{z}$)) : } 

\[ \max \sum_{k \in K}\alpha_k - \sum_{i \in H}\sum_{k \in K} \hat{z}_iu_{ik}\]
tel que
\begin{subequations}
    \begin{align}
        \alpha_k - u_{e_1k} - u_{e_2k} \le F_{ek}, \quad &\forall{k\in K}, \forall{e \in E}, |e| = 2&\\
        \alpha_k - u_{e_1k} \le F_{ek}, \quad &\forall{k\in K}, \forall{e \in E}, |e| = 1&\\
        u_ik \ge 0, \quad &\forall{i \in H}, \forall{k \in K}
    \end{align}
\end{subequations}

où $\alpha_k, k\in K$ sont les variables associées aux contraintes \ref{2a} et $u_{ik}, i \in H, k \in K$ les variables associées aux contraintes \ref{2c}.


Soit D l'ensembles des solutions admissibles de SD et soit $P_D$ l'ensemble des points extrêmes de D. On remarque que D n'est pas modifié si on modifie $\hat{z}$ et puisque $F_{ek} \geq 0$ pour tout $e \in E_k$ and $k \in K$, le vecteur nul est toujours solution du sous-problème dual. En conséquence, par dualité forte, soit le sous-problème primal a une solution et est borné, soit il n'a pas de solution.

Comme pour tout vecteur binaire z tel que $\sum_{i \in H} z_i \ge 1$, les sous-problèmes primal et dual possèdent une solution et sont bornés, la fonction objectif duale s'écrit :
\[ \max_{(\alpha,u) \in P_D} \sum_{k \in K}\alpha_k - \sum_{i \in H}\sum_{k \in K} \hat{z}_iu_{ik}\]

De plus, les coupes de faisabilité associées au rayons extrêmes de l'ensemble D ne sont pas nécessaires du fait de la contrainte $\sum_{i \in H} z_i \ge 1$. 




%\smallskip
\paragraph{Problème maître de Benders (PM) :} En introduisant la variable supplémentaire $\eta$ représentant le coût global de transport, les auteurs aboutissent à un problème mixte avec card(H) variables binaires $z_i$ et une variable continue $\eta$. Ce :  


\[ \min \sum_{i \in H} f_iz_i + \eta\]
tel que
\begin{subequations}
    \begin{align}
       \eta \ge \sum_{k \in K}\alpha_k - \sum_{i \in H}\sum_{k \in K} z_iu_{ik}, \quad &\forall{(\alpha,u) \in P_D}&\\
       \sum_{i \in H} z_i \ge 1 \quad &&\\
        z_i \in \{0,1\}, \quad & \forall{i \in H}&
    \end{align}
\end{subequations}
	

Les contraintes (4a) représentent les coupes d'optimalité.

