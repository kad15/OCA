\section{Modèle complet}

En partant du modèle de Hamacher et al. \cite[Adapting polyhedral properties from facility to hub location problems (2004)]{hln}, les auteurs ont défini un ensemble $E_k$ d'arêtes $e$ d'extrémités identiques ou différentes, extrémités qui sont candidates à devenir des hub pour chaque bien k. $E_k$ réduit le choix des hubs potentiels contenu dans H d'où le modèle amélioré suivant qui exploite les propriétés spécifiques des solutions optimales du problème UHLPMA : 

\[ \min \sum_{i \in H}f_iz_i + \sum_{k \in K}\sum_{e \in E_k}F_{ek}x_{ek}\]
tel que
\begin{subequations}
    \begin{align}
        \sum_{e \in E_k}x_{ek} = 1, \quad &\forall{k\in K}&\\
        \sum_{e \in E_k:i\in e}x_{ek} \le z_i,\quad &\forall{i \in H}, \forall{k\in K}&\\
        x_{ek} \ge 0, \quad &\forall{k \in K}, \forall{e \in E_k}&\\
        z_i\in \{0,1\},\quad &\forall{ i \in H}&
    \end{align}
\end{subequations}
 où $H$ est l'ensemble des localisations possibles pour un hub, $K$ est l'ensemble des biens qui doivent être acheminés, $f_i$ est le coût fixe de transformation en hub d'un noeud $i\in H$, $z_i$ est une variable binaire égale à 1 si le noeud $i$ est un hub, et à 0 sinon, $F_{ek}$ est le coût de transport non orienté pour un arc du graphe $e \in H\times H$ et un bien $k \in K$, et $x_{ek}$ est un variable binaire égale à 1 si le bien $k \in K$ transite par l'arc $e \in H\times H$ et à 0 sinon.
 
 Le coût de transport non orienté pour un arc $e = (i,j) \in H\times H$ est le minimum du coût de transport dans le sens $i \rightarrow j$ et du coût de transport dans le sens $j \rightarrow i$, c'est à dire $F_{ek} = \min \{ \hat{F}_{ijk};\hat{F}_{jik}\}$.


