\section{Grandes lignes des résultats obtenus}

Les expérimentations numériques sur des instances de 25 à 200 noeuds montrent que les algorithmes \emph{1-cut} et \emph{$|H|$-cut} proposés par l'article résolvent la plupart des instances, même les plus difficiles avec 200 noeuds. On constate également que l'algorithme  \emph{$|H|$-cut} est plus performant que le \emph{1-cut}, car il résoud plus d'instances, avec un temps d'éxécution plus court et un nombre d'itérations moins élevé.

Les stratégies pour implémenter des coupes sont aussi testées, et on voit que la stratégie SC utilisant les coupes produites par l'algorithme 4 proposé par les auteurs est plus performantes que les stratégies POC utilisant des coupes Pareto-optimales et NC utilisant les coupes produites par l'algorithme 2 de l'article.

Les expérimentations numériques permettent également de démontrer l'efficacité de l'heuristique développée par les auteurs, car cette heuristique trouve très souvent la solution optimale, même sur de grandes instances, et ce avec un temps d'éxécution très inférieur (par un facteur 20 sur les plus grandes instances avec 200 noeuds). On voit aussi que la solution donnée par l'heuristique permet de générer de bonnes coupes, qui améliorent les résultats de l'algorithme exact. 

Ensuite, les auteurs montrent que les tests d'élimination qu'ils ont développés sont efficaces sur les grandes instances. En particulier, il est montré que le plus efficace est d'implémenter une coupe induite par la solution heuristique et les deux tests d'élimination proposés dans la section 4.3. Sur les plus grandes instances de taille 200, le temps d'éxécution est 73\% du temps requis par l'algorithme implémenté sans aucune coupe ni aucun test. 

Les auteurs présentent également une comparaison de la meilleure version de leur décomposition de Benders avec plusieurs algorithmes antérieurs fournissant une solution exacte. Les résultats de cette expérience montrent que la décomposition de Benders est beaucoup plus performantes que ces algortihmes : elle permet de résoudre les problèmes de grande taille beaucoup plus vite. On remarque également que les algorithmes auxquels la décomposition de Benders est comparée rencontrent des problèmes de mémoire ou de limitation de temps de résolution sur les instances de plus de 100 noeuds, tandis que la décomposition de Benders résoud toute les instances proposées.

Enfin, pour tester la méthode heuristique et l'algorithme de décomposition de Benders, les auteurs proposent un nouveau set d'instances de très grande taille allant jusqu'à 500 noeuds. Même sur ces instances, la procédure heuristique et le meilleur algorithme de décomposition de Benders trouvent la solution optimale dans la majorité des cas. On remarque que le temps CPU devient très important, avec plus de une heure pour l'heuristique et plus de 9 heures pour la décomposition de Benders. Cependant, étant donné la difficulté de telles instances, ces temps d'éxécution ne sont pas disqualifiant, et le seul fait de résoudre la majorité des instances montre l'efficacité des algorithme proposés.
