\section{Problématique générale}

Le problème de localisation traité dans l'article est connu sous l'appellation \textit{Uncapacitated Hub Location Problem with Multiple Assignments} (UHLPMA). Il est NP-difficile donc peu tractable pour des grandes instances. L'idée générale est de transporter dans un réseau des entités/marchandises d'une origine O vers une destination D de manière la plus efficace possible. L'objectif est donc de choisir noeuds, par exemple des aéroports, qui vont devenir des hubs et de déterminer les flux des marchandises dans le réseau de manière à minimiser les coûts fixes et les coûts variables de transport de marchandises en mettant en place des hubs, choisis parmi les noeuds du réseau, et générateurs d'économies d'échelle. 

\paragraph{Description du problème UHLPMA}
Soit un graphe orienté complet G(N,A) constitué de N noeuds représentant des centres d'activité tel que des aéroports et de A arcs. H $\subseteq $ N représente l'ensemble des n\oe uds potentiellement transformables en hubs\footnote{L'ensemble des hubs constitue un graphe totalement connecté.}. On ne connaît pas à l'avance le nombre de ces hubs, uniquement leurs coûts d'installation. La capacité des hubs et les flux ne sont pas bornés. En outre, le caractère \textit{Multiple Assignement} fait référence au fait que chaque centre Origine/Destination (O/D) peut alimenter/être alimenté par plusieurs hub.
K représentera l'ensemble des marchandises dont les noeuds d'origine et de destination
appartiennent à N. $ W_k $ désignera la quantité de bien k $\in$ K  à transporter d'une origine o(k) $\in$ N vers une destination d(k) $\in$ N. Les coûts fixes de transformation d'un n\oe ud i $\in$ H en hub seront notés $ f_i$. Les coûts variables de transport, ou distance entre deux n\oe uds $i$ et $j$, seront désignés par $d_{ij}$. Le coût de transport d'un bien k le long du chemin (o(k),i,j,d(k)) est noté $ \hat{F}_{ijk}$. Un chemin entre une origine et une destination ne pourra contenir que 1 ou 2 hubs car ces derniers sont totalement connectés et le coût $d_{ij}$ est supposé vérifier l'inégalité triangulaire. 




  
















%\cite[p. 2] {ccl}


%\subsection{}
%\paragraph{}

%\subsection{Description du système et de son environnement}
%\begin{figure}[H]
%	\begin{center}	
%		\includegraphics[scale=0.3]{images/0/ctx}
%		\caption{Environnement du système}
%		\label{ctx}
%	\end{center}
%\end{figure}


 