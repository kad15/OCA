\section{Problématique générale}

Le problème de localisation traité dans l'article est connu sous l'appellation \textit{Uncapacitated Hub Location Problem with Multiple Assignments} (UHLPMA). Il est NP-difficile donc peu tractable pour des grandes instances. L'idée générale est de transporter dans un réseau des entités/marchandises d'une origine O vers une destination D de manière la plus efficace possible. L'objectif est donc de minimiser les coûts fixes et les coûts variables de transport de marchandises en mettant en place des hubs, choisis parmi les noeuds du réseau, et générateurs d'économies d'échelle. 


\paragraph{Description du problème UHLPMA}
Soient un graphe orienté G(N,A) constitué de N noeuds représentant des centres d'activité tel que des aéroports et de A arcs et H $\subseteq $ N un sous-ensemble de candidats destinés potentiellement à devenir des Hubs. On ne connaît pas à l'avance le nombre de ces hubs, uniquement leurs coûts d'installation. La capacité des hubs et les flux ne sont pas bornés. En outre, le caractère \textit{Multiple Assignement} fait référence au fait que chaque centre Origine/Destination (O/D) peut alimenter/être alimenté par plusieurs hub.
K représentera l'ensemble des marchandises dont les noeuds d'origine et de destination
appartiennent à N. $ W_k $ désignera la quantité de marchandise k $\in$ K  à transporter d'une origine o(k) $\in$ N vers une destination d(k) $\in$ N.



  
















%\cite[p. 2] {ccl}


%\subsection{}
%\paragraph{}

%\subsection{Description du système et de son environnement}
%\begin{figure}[H]
%	\begin{center}	
%		\includegraphics[scale=0.3]{images/0/ctx}
%		\caption{Environnement du système}
%		\label{ctx}
%	\end{center}
%\end{figure}


 