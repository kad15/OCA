\section{Techniques d’amélioration de l'algorithme}


\paragraph{Reformulation multicoupe : }
Cette technique décompose le problème maître en $|H|$ sous-problèmes indépendants, un pour chaque n\oe ud. Seules $|H|$ coupes d'optimalité sont générées à partir de chaque polyèdre dual associé au sous-problème $j$ contre $|K|$ si la décomposition complète des sous-problèmes duaux avait été utilisée. Auquel cas, les $|K|$ coupes par itération auraient été contre productives en terme de temps de calcul.


\[ \min \sum_{i \in H} f_iz_i + \sum_{i \in H} \eta_i  \]
tel que
\begin{subequations}
	\begin{align}
	\eta_j \ge \sum_{k \in K_j}\alpha_k - \sum_{i \in H}\sum_{k \in K_j} z_iu_{ik} \quad &\forall j \in H, \forall{(\alpha,u) \in P_{D^j}}&\\
	\sum_{i \in H} z_i \ge 1 \quad &&\\
	z_i \in \{0,1\}, \quad & \forall{i \in H}&
	\end{align}
\end{subequations}

\paragraph{Coupe Pareto-optimale : } Les auteurs ont utilisé ici la méthode de Magnanti et Wong \cite[1981]{mw} pour obtenir des coupes non dominées plus efficaces et ainsi améliorer la convergence.

\paragraph{Tests d'élimination : }
Ici, l'efficacité et la convergence de l'algorithme de décomposition de Benders sont améliorées en réduisant la taille du modèle original. Les auteurs ont utilisés deux tests
visant à éliminer des n\oe ud de l'ensemble $H$ dont on sait, compte tenu des informations acquises lors des itérations, qu'elles ne pourront pas apparaître dans la solution optimale à la manière d'un Branch\&Bound.


\paragraph{Heuristique : }L'heuristique proposée par les auteurs est constituée de deux phases : une phase d'\textbf{estimation} et une phase d'\textbf{intensification}. La phase d'estimation construit itérativement un ensemble de solutions admissibles qui servent à calculer un intervalle sur le nombre de hubs ouverts.
L'heuristique d'intensification génère des solutions admissibles contenant des ensembles de hubs ouverts dont la cardinalité est située dans l'intervalle précédent.
Les deux phases utilisent une \textbf{procédure de construction commune} qui construit aléatoirement une solution admissible de nombre de hubs ouverts déterminé et l'améliore
à l'aide d'une procédure de recherche locale. Cette dernière utilise 3 types de solutions voisines : avec ajout d'un hub à la solution courante, avec fermeture d'un hub et enfin avec ouverture d'un hub et fermeture d'un autre.